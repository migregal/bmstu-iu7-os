\chapter{Используемые структуры}

В данной лабораторной работе в качестве операционной системы использовалась \texttt{Manjaro Linux 21.2.26}

\listingfile{fs.h}{fsh}{C}{Листинг структуры \texttt{filename} (\texttt{include/linux/fs.h})}{}

\listingfile{internal.h}{internalh}{C}{Листинг структуры \texttt{open\_flags} (\texttt{fs/internal.h})}{}

\listingfile{namei.c}{namec}{C}{Листинг структуры \texttt{nameidata} (\texttt{fs/namei.c})}{}

\section*{Флаги системного вызова \texttt{open()}}

\texttt{O\_EXEC} --- открыть только для выполнения (результат не определен, при открытии директории).

\texttt{O\_RDONLY} --- открыть только на чтение.

\texttt{O\_RDWR} --- открыть на чтение и запись.

\texttt{O\_SEARCH} --- открыть директорию только для поиска (результат не определен, при использовании с файлами, не являющимися директорией).

\texttt{O\_WRONLY} --- открыть только на запись.

\texttt{O\_APPEND} --- файл открывается в режиме добавления, перед каждой операцией записи файловый указатель будет устанавливаться в конец файла.

\texttt{O\_CLOEXEC} --- включает флаг \texttt{close-on-exec} для нового файлового дескриптора, указание этого флага позволяет программе избегать дополнительных операций fcntl \texttt{F\_SETFD} для установки флага \texttt{FD\_CLOEXEC}.

\texttt{O\_CREAT} --- если файл не существует, то он будет создан.

\texttt{O\_DIRECTORY} --- если файл не является каталогом, то open вернёт ошибку.

\texttt{O\_DSYNC} --- файл открывается в режиме синхронного ввода-вывода (все операции записи для соответствующего дескриптора файла блокируют вызывающий процесс до тех пор, пока данные не будут физически записаны).

\texttt{O\_EXCL} --- если используется совместно с \texttt{O\_CREAT}, то при наличии уже созданного файла вызов завершится ошибкой.

\texttt{O\_NOCTTY} --- если файл указывает на терминальное устройство, то оно не станет терминалом управления процесса, даже при его отсутствии.

\texttt{O\_NOFOLLOW} --- если файл является символической ссылкой, то open вернёт ошибку.

\texttt{O\_NONBLOCK} --- файл открывается, по возможности, в режиме non-blocking, то есть никакие последующие операции над дескриптором файла не заставляют в дальнейшем вызывающий процесс ждать.

\texttt{O\_RSYNC} --- операции записи должны выполняться на том же уровне, что и \texttt{O\_SYNC}.

\texttt{O\_SYNC} --- файл открывается в режиме синхронного ввода-вывода (все операции записи для соответствующего дескриптора файла блокируют вызывающий процесс до тех пор, пока данные не будут физически записаны).

\texttt{O\_TRUNC} --- если файл уже существует, он является обычным файлом и заданный режим позволяет записывать в этот файл, то его длина будет урезана до нуля.

\texttt{O\_LARGEFILE} --- позволяет открывать файлы, размер которых не может быть представлен типом off\_t (long).

\texttt{O\_TMPFILE} --- при наличии данного флага создаётся неименованный временный файл.

\chapter{Схемы алгоритмов}

\imgw{open}{h!}{0.8\textwidth}{Схема работы функции open}

\imgw{build_open_flags}{h!}{0.7\textwidth}{Схема работы функции build\_open\_flags}

\imgw{getname_flags}{h!}{\textwidth}{Схема работы функции getname\_flags}

\imgw{alloc_fd}{h!}{\textwidth}{Схема работы функции alloc\_fd}

\imgw{path_openat}{h!}{\textwidth}{Схема работы функции path\_openat}

\imgw{open_last_lookups}{h!}{0.9\textwidth}{Схема работы функции open\_last\_lookups}

\imgw{lookup_open}{h!}{0.9\textwidth}{Схема работы функции lookup\_open}

\imgw{set_nameidata}{h!}{0.6\textwidth}{Схема работы функции set\_nameidata}

\imgw{restore_nameidata}{h!}{0.6\textwidth}{Схема работы функции restore\_nameidata}